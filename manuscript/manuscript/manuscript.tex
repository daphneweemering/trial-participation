\documentclass[preprint, 3p,
authoryear]{elsarticle} %review=doublespace preprint=single 5p=2 column
%%% Begin My package additions %%%%%%%%%%%%%%%%%%%

\usepackage[hyphens]{url}

  \journal{Elsevier journal} % Sets Journal name

\usepackage{lineno} % add

\usepackage{graphicx}
%%%%%%%%%%%%%%%% end my additions to header

\usepackage[T1]{fontenc}
\usepackage{lmodern}
\usepackage{amssymb,amsmath}
\usepackage{ifxetex,ifluatex}
\usepackage{fixltx2e} % provides \textsubscript
% use upquote if available, for straight quotes in verbatim environments
\IfFileExists{upquote.sty}{\usepackage{upquote}}{}
\ifnum 0\ifxetex 1\fi\ifluatex 1\fi=0 % if pdftex
  \usepackage[utf8]{inputenc}
\else % if luatex or xelatex
  \usepackage{fontspec}
  \ifxetex
    \usepackage{xltxtra,xunicode}
  \fi
  \defaultfontfeatures{Mapping=tex-text,Scale=MatchLowercase}
  \newcommand{\euro}{€}
\fi
% use microtype if available
\IfFileExists{microtype.sty}{\usepackage{microtype}}{}
\usepackage[]{natbib}
\bibliographystyle{plainnat}

\ifxetex
  \usepackage[setpagesize=false, % page size defined by xetex
              unicode=false, % unicode breaks when used with xetex
              xetex]{hyperref}
\else
  \usepackage[unicode=true]{hyperref}
\fi
\hypersetup{breaklinks=true,
            bookmarks=true,
            pdfauthor={},
            pdftitle={Barriers and facilitators of clinical trial participation in neurodegenerative diseases: A systematic review and meta-analysis},
            colorlinks=false,
            urlcolor=blue,
            linkcolor=magenta,
            pdfborder={0 0 0}}

\setcounter{secnumdepth}{5}
% Pandoc toggle for numbering sections (defaults to be off)


% tightlist command for lists without linebreak
\providecommand{\tightlist}{%
  \setlength{\itemsep}{0pt}\setlength{\parskip}{0pt}}






\begin{document}


\begin{frontmatter}

  \title{Barriers and facilitators of clinical trial participation in
neurodegenerative diseases: A systematic review and meta-analysis}
    \author[Neuro]{Daphne N. Weemering%
  %
  }
  
    \author[Rehab, Excel]{Anita Beelen%
  %
  }
  
    \author[Neuro]{Tessa Kliest%
  %
  }
  
    \author[Neuro]{Lucie A.G. van Leeuwen%
  %
  }
  
    \author[Neuro]{Leonard H. van den Berg%
  %
  }
  
    \author[Neuro, Julius]{Ruben P.A. van Eijk%
  \corref{cor1}%
  }
   \ead{r.p.a.vaneijk-2@umcutrecht.nl} 
      \affiliation[Neuro]{Department of Neurology, UMC Utrecht Brain
Centre, University Medical Centre Utrecht, Utrecht, the Netherlands.}
    \affiliation[Rehab]{Department of Rehabilitation, Physical Therapy
Science \& Sports, UMC Utrecht Brain Centre, University Medical Centre
Utrecht, the Netherlands}
    \affiliation[Excel]{Centre of Excellence for Rehabilitation
Medicine, UMC Utrecht Brain Centre, University Medical Centre Utrecht,
and De Hoogstraat Rehabilitation, Utrecht, the Netherlands}
    \affiliation[Julius]{Biostatistics \& Research Support, Julius
Centre for Health Sciences and Primary Care, University Medical Centre
Utrecht, Utrecht, the Netherlands}
    \cortext[cor1]{Corresponding author}
  
  \begin{abstract}
  \textbf{Background}: Study participants in clinical trials for
  neurodegenerative disorders are selective and underrepresentative for
  all patients with the disease. This limits generalizability of study
  results and makes the safety and efficacy of interventions unknown for
  the majority of patients. In this study, we aim to identify barriers
  and facilitators for clinical trial participation. These identified
  factors inform recommendations for the inclusion of the broader
  population.

  \textbf{Methods}: We conducted a systematic literature search of
  quantitative and qualitative articles reporting barriers and
  facilitators of clinical trials for neurodegenerative diseases.
  Studies evaluating participating in non-interventional research were
  excluded. Barriers and facilitators were extracted; the frequency with
  which patients identified barriers/facilitators were pooled across
  studies and summarized as proportions.

  \textbf{Results}: Thirty-six unique studies were included, enrolling a
  total of 5113 patients and/or people related to Alzheimer's disease
  (n=1261), Parkinson's disease (n=2789), Huntington's disease (n=696)
  or amyotrophic lateral sclerosis (n=367). The most commonly mentioned
  barriers/facilitators were XXX.
  \end{abstract}
    \begin{keyword}
    Trial participation \sep 
    neurodegenerative diseases
  \end{keyword}
  
 \end{frontmatter}

\hypertarget{introduction}{%
\section{Introduction}\label{introduction}}

Elit sociosqu semper nascetur; magna tellus facilisis potenti inceptos.
Nibh ut lacinia conubia senectus; fames ullamcorper tempus mattis.
Eleifend pretium scelerisque fusce a dignissim etiam, class torquent
imperdiet cursus. Elementum venenatis consequat cubilia, tempus libero
auctor, habitant, lectus a pharetra. Rutrum penatibus class nisi porta
dapibus nec ad suspendisse ullamcorper, justo conubia vulputate nibh
habitant etiam tempor, varius accumsan.

Consectetur urna praesent nam ullamcorper class habitant facilisi taciti
pharetra imperdiet convallis fames. Faucibus eros porttitor sapien mus
vehicula mauris, at arcu elementum enim viverra? Neque facilisis taciti
enim cras libero per nascetur. Ut varius dui euismod blandit luctus
maecenas class quis lacinia penatibus! Hac integer fames!

Lorem sodales potenti odio cursus blandit, ac venenatis pharetra non
tortor. Faucibus commodo mus dis augue a. Risus risus natoque primis
montes aliquam purus fermentum litora potenti! At at platea, facilisis,
vulputate sociosqu tellus at? Iaculis rhoncus litora, curabitur cras
netus cursus nam arcu. Penatibus turpis faucibus aliquet donec dui,
penatibus dictumst, in luctus, nulla sodales -- lectus ornare nascetur
gravida ante vel sed pretium nostra.

Adipiscing suspendisse torquent maecenas per, class tempus mus vitae
pharetra fermentum. Imperdiet fusce suspendisse hac turpis, urna in
penatibus nostra aptent? Conubia lectus justo augue nam enim pretium
quam quam donec. Hendrerit ante vitae laoreet enim eu lobortis vel
habitant in. Orci nisl nostra convallis ultrices varius at faucibus
ornare eget lectus. Curae arcu ante vivamus habitasse nostra;
condimentum fusce suspendisse urna. Quisque aptent feugiat tristique
potenti conubia mauris placerat accumsan sodales. Congue sollicitudin
mauris metus phasellus sollicitudin rhoncus, scelerisque sodales nam!

Amet non sociosqu, condimentum eu -- lacinia nisl facilisis, laoreet
gravida habitasse phasellus. Enim interdum tellus: euismod et montes
potenti netus quisque vehicula magnis. Purus facilisis lobortis
porttitor fames penatibus nascetur leo? Platea litora porta est, cum:
massa malesuada purus arcu. Aliquam nibh est id nec felis tempor mollis
himenaeos tristique. Lacinia elementum nisl augue porta ante quis
hendrerit, dui ornare, mollis, accumsan laoreet tristique nascetur etiam
egestas nec facilisis.

\hypertarget{methods}{%
\section{Methods}\label{methods}}

\hypertarget{search-strategy}{%
\subsection{Search strategy}\label{search-strategy}}

The aim of the systematic review was to identify all original research
articles, both qualitative and quantitative, that report on barriers and
facilitators to participation in clinical trials. Studies were
identified in the PubMed and EMBASE databases, as well as by screening
reference lists from relevant reviews. The search was limited to the
most common neurogenerative diseases: Alzheimer's disease (AD),
Parkinson's disease (PD), Huntington's disease (HD) and amyotrophic
lateral sclerosis (ALS) \citep{hou2019, zahra2020}. Search terms
included ``barriers'', ``facilitators'', ``clinical trials'',
``participation'', ``Alzheimer's disease'', ``Parkinson's disease'',
``Huntington's disease'', and ``amyotrophic lateral sclerosis'', and
their synonyms (\textbf{Supplement 1}). The search was discussed with an
information specialist and was limited to qualitative and quantitative
studies written in English and Dutch. The final reference list was
generated in January 2022.

\hypertarget{study-selection}{%
\subsection{Study selection}\label{study-selection}}

After removal of duplicates, the reference list was analyzed using
Automated Systematic Review (ASReview, version v0.17rc0) -- a machine
learning framework which labels and subsequently ranks titles and
abstracts based on the relevance of the key terms entered into the
software \citep{vandeschoot2021}. Relevant key terms were:
``facilitator'', ``barrier'', ``participate'', ``refusal'',
``perspective'', ``patient selection'', ``retention'', ``attitude'',
``recruit'', ``enroll'', ``accrual'', ``attrition'', and ``clinical
trial''. ``Animal'' and/or ``mouse'' were marked as irrelevant key
terms.

The first 200 ranked abstracts were screened by two reviewers (TK and
LAGvL) and were discussed until consensus was reached
(\textbf{Supplement 2.1}). The following abstract were screened by one
reviewer (TK). References were eligible for full-text screening if the
abstract reported facilitators and barriers to participation in clinical
trials. Full text articles had to meet the following criteria to be
using in the thematic and/or meta-analysis (\textbf{Supplement 2.2}):
original research published in a peer reviewed journal, reporting
barriers and/or facilitators for (a) clinical trial(s) with a drug
therapy, concerning patients with AD, PD, HD, or ALS. Qualitative
studies were assessed by using the Critical Appraisal Skills Program
(CASP- quality assessment tool \citep{casp2018}.

\hypertarget{data-extraction}{%
\subsection{Data extraction}\label{data-extraction}}

Data on facilitators and barriers were independently extracted from the
articles by TK and LAGvL with a pre-prepared extraction scheme. The
first ten articles and the corresponding study extraction schemes were
discussed in depth and compared between TK and LAGvL. Once consensus was
reached, data of interest were extracted independently. Studies solely
reporting demographics, participants at risk of neurodegenerative
diseases, outcomes represented by \(\leq 5\%\) of the sample size, or
articles where the patient population with neurodegenerative disease was
\(\leq 50\%\) of the total sample size (e.g., studies including various
disease populations) were excluded from the analysis.

All barriers and facilitators for trial participation were reported from
patients, caregivers and/or health care professionals (HCPs). The
identified barriers and facilitators were clustered in three overarching
themes: patient related-, study procedures related- and HCP related
factors regarding (not) participating. The study procedures related- and
HCP related factors were considered modifiable -- factors that can be
applied to and amended before and during clinical trial. The patient
related factors were appraised as motivators and beliefs -- intrinsic
factors that cannot be amended during the design phase of a trial to
improve enrollment. After all barriers and facilitators were collected
and grouped, TK did a final review to identify whether the created
themes and subthemes represented the reported barriers and facilitators
identified in the analyzed articles. Final consensus on the (sub)themes
was reached during an in-depth meeting with RPAvE, AB, DW and TK.

\hypertarget{statistical-analysis}{%
\subsection{Statistical analysis}\label{statistical-analysis}}

The quantitative data (i.e., the proportion of patients reporting on
barriers/facilitators in closed interview questions/surveys) were
synthesized using meta-analyses to obtain overall effect size estimates
for each of the modifiable barriers and facilitators. The meta-analyses
were performed exclusively on patient-reported barriers and
facilitators.

The proportions for each barrier/facilitator were logit transformed
prior to pooling. Random intercept logistic regression models were
fitted, accounting for the binomial structure of the data and to account
for the between-study heterogeneity \citep{stijnen2010}. Proportions and
\(95\%\) Clopper-Pearson (i.e., exact binomial confidence interval)
confidence intervals \citep{clopper1934} are reported for each
barrier/facilitator. To give a good indication of the variance between
the studies, the following heterogeneity statistics were provided: the
between study variance parameter \(\tau^2\), the Q-statistic, and the
I2- and H2-statistics, the latter two with a \(95\%\) confidence
interval.

Heterogeneity between studies could be expected due to the different
disease populations and the different types of responders (i.e.,
patients, caregivers, HCPs or a combination of these). Therefore,
subgroup analyses were performed on each barrier/facilitator, with the
disease populations and the responder type as subgroups. We fitted
mixed-effects model, assuming the subgroups to be fixed and the studies
within the subgroups to be random. Cochran's Q \citep{cochran1954} was
used to determine whether overall differences in effect size were
significant. We used the \(I^2\) statistic \citep{higgins2002} to
determine between study heterogeneity. P-values are reported with
decimals -- values smaller or around 0.05 are considered statistically
significant. All quantitative analyses are performed in R (version 4.2.1
(2022-06-23)) \citep{Rmanual} with the `meta' package \citep{meta}.

\hypertarget{results}{%
\section{Results}\label{results}}

In total, 560 abstracts were reviewed (AD: n = 206, PD: n = 177, HD: n =
69, ALS: n = 108), resulting in the inclusion of 85 abstracts (AD: n =
61, PD: n = 15, HD: n = 4, ALS: n = 5) for full text scrutinization. A
Cohen's kappa of 0.86 was established between the reviewers, indicating
an almost perfect agreement for abstract review of 200 articles. Of the
85 full text articles, 49 articles were excluded based on the following
reasons: the article did not include participants with a diagnosis
(e.g., people at risk or health elderly (n = 23)), did not assess
barriers or facilitators for participation in drug therapy clinical
trials (n = 23), and were not an original research article (n = 3; see
\textbf{figure 1} for the study selection flow diagram). Thus, we
included 36 full text articles (AD: n = 17, PD: n = 12, HD: n = 4, ALS:
n = 2, PD \(+\) ALS: n = 1) for the final data extraction. No
qualitative studies were excluded based on the results of the CASP
(\textbf{Supplement 3}).

\hypertarget{study-characteristics}{%
\subsection{Study characteristics}\label{study-characteristics}}

\textbf{Table 1} shows the main characteristics of the included studies.
Two articles {[}\textbf{REFS}{]} included both qualitative and
quantitative study methods; one article {[}\textbf{REF}{]} included two
different qualitative methods and one article {[}\textbf{REF}{]}
included both people with PD and ALS. All methods and patient
populations were separately analyzed and reported. Finally, the review
resulted in the inclusion of 25 quantitative studies (survey/structured
interview: n = 4582) and 15 qualitative studies (open-ended
questions/focus groups/semi-structured interviews: n = 531). Of the
included studies, 19 (48\%) focuses on AD, 14 (35\%) on PD, 4 (10\%) on
HD, and 3 (8\%) on ALS. Per disease, data was collected of 1261 people
with AD, 2789 people with PD, 696 people with HD, and 367 people with
ALS. Thus, a total of 5113 people with/related to neurodegenerative
diseases were included.

Of the included studies, 65\% was conducted in the United States (US).
The remainder of the studies was conducted in Sweden (8\%), Australia
(5\%), Finland (5\%), and the United Kingdom (UK), Israel, Europe
(several countries), Singapore, US together with UK, and Canada (the
latter countries all occurred once). The focus of the studies also
varied in terms of perspective and data collection method. 19 studies
(48\%) focused solely on the perspective of the patient. 8 studies
(20\%) included the perspective of both patients and caregivers, the
perspective of the caregivers was incorporated in 7 studies (18\%), 4
studies (10\%) included patients, caregivers, and healthcare
professionals, and 2 studies (5\%) investigated healthcare
professionals' perspectives on factors influencing trial participation.
Regarding the data collection method, 24 studies (60\%) used a survey,
11 studies (28\%) collected data using semi-structured interviews, 2
studies (5\%) used open-ended question, 2 other studies conducted focus
groups, and 1 study (3\%) used a structured interview to collect data.

\hypertarget{thematic-analysis-of-qualitative-data}{%
\subsection{Thematic analysis of qualitative
data}\label{thematic-analysis-of-qualitative-data}}

Several themes and subthemes corresponding to barriers, facilitators,
motivators, and beliefs for trial participation were identified in the
studies. These (sub)themes are displayed in \textbf{figure 2}.

\hypertarget{meta-analyses-of-quantitative-data}{%
\subsection{Meta-analyses of quantitative
data}\label{meta-analyses-of-quantitative-data}}

The meta-analyses of the proportional data are displayed in
\textbf{figure 3}. The primary barrier/facilitator for engaging in
clinical trials under patients with a neurodegenerative disease, was
awareness (\(N_{studies}\) = 1, estimate (95\% CI) = 78.5\% (75.6 to
81.2)). Second and third ranked barriers/motivators for trial
participation were the relationship with clinical staff (\(N_{studies}\)
= 6, estimate (95\% CI) = 75.0\% (58.8 to 86.3)) and placebo/sham use
(\(N_{studies}\) = 5, estimate (95\% CI) = 71.2 (21.3 to 95.8)).

Due to synthesizing data of different diseases and different types of
responders, between-study heterogeneity was expected a priori, and was
found for each barrier/facilitator. The heterogeneity statistics are
reported in \textbf{table 2}. The Q-statistics were significantly
different from zero (\(p < 0.001\)), the \(I^2\) statistics also mark
substantial heterogeneity (\textgreater90\%).

\hypertarget{discussion}{%
\section{Discussion}\label{discussion}}

\renewcommand\refname{References}
\bibliography{mybibfile.bib}


\end{document}
